\section{Introduction}
\label{Sec:Introduction}

% what is link prediction problem? what is the usages of link prediction?

The goal of link prediction in a social network graph \cite{liben2007link} is to estimate the likelihood of the relationship between two nodes in future, based on the observed network. Recommending such future links have multiple applications such as friendship, item, or ad suggestions, network completion, or predicting protein-protein interactions.

% challenges in heterogeneous link prediction, what is not covered? temporal and hero?

Traditional link prediction techniques  consider social networks to be homogeneous, i.e., graphs with only one type of edges and nodes, however, most real-world social networks (e.g. Twitter, Facebook, DBLP) are heterogeneous, i.e., have multiple relation and node types. For example, in a bibliographic social network there are different types of nodes such as authors, papers, and venues, and edges such as write, cite and publish. There are limited number of works that focused on this problem. For example, the probabilistic latent tensor factorization model... Recent works, such as \cite{sun2011pathsim}, investigated this problem. However, such techniques do not consider the dynamics of social networks and ignore the timestamps associated to the relations. 
%\cite{Zhu2016} \cite{sun2011pathsim} \cite{Sun:2012:HRP:2124295.2124373}  \cite{huang2016meta} \cite{wang2016relsim} \cite{sun2013pathselclus} \cite{sun2011ASONAM} \cite{Yang2012} \cite{liben2007link}


In this work we study the problem of temporal and heterogeneous link prediction, that can be stated as follows: \textit{Given a dynamic heterogeneous social network graph G (network with different types of nodes and links, attached with timestamps), how can we predict the future graph structure?}





\subsection{Motivation}

\amin{Motivate the problem by a real example and show the points made in this para.}

The link prediction problem for homogeneous networks has been studied in the past. However most real social networks are heterogeneous and relations between different entities have different semantic meanings. Thus techniques for homogeneous networks can not be directly applied on heterogeneous ones. Recent works, such as \cite{sun2011pathsim,sun2011ASONAM}, investigated this problem. However, such techniques do not consider the dynamics of social networks and ignore the timestamps associated to the relations. This is important as incorporating temporal changes helps in more accurate prediction (e.g. \cite{Zhu2016}). On the other hand, previous work on temporal link prediction scarcely studied heterogeneousness of social networks.To the best of our knowledge, the problem of link prediction for dynamic (temporal) heterogeneous networks was not studied before.



\subsection{Contributions}

The main contributions of our work include:

\begin{itemize}

\item We present a technique, called \texttt{LinkPredict}, that predict links between two nodes of given types and a target relation;

\item An evaluation of the accuracy and performance of the proposed algorithm on real social network data.

\end{itemize}






