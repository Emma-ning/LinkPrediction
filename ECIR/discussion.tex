\section{Discussion}

%Increasing accuracy.... We do not claim that the linear combination is the best ... After $p$-value analysis some latent features might be correlated with meta paths. Removing may increase the accuracy or we can remove those with lower $p$-value. This needs careful analysis as it might be dependent to number of intervals or the size of latent feature.


\subsubsection{Feature Engineering.}  Latent features are shown to be more predictive compared to unsupervised scoring techniques such as Katz, Adamic, and Preferential Attachment \cite{menon2011link,Zhu2016}. In our experiments we observed that combining latent with meta path-based features can increase prediction accuracy. However, if latent features learn similar structure as topological features do, then mixing them may not be beneficial. In such cases feature engineering techniques can be applied. 

%As explained in \cite{menon2011link}, one may also augment the model by incorporating some information regarding node affinities using implicit/explicit attributes and define node features.

%In this work, in this work we observed that meta path-based features significantly help in 

%Experiments in \cite{menon2011link} shows combining the latent structure and side-information increases the prediction accuracy.



%In this work we modelled the predicted graph $ \hat{G}_t(i,j)$ as a combination of meta path features and latent features $\Phi(z_{i}^Tz_{j} + f_D(z_{i,j};w))$. 
%As explained in \cite{menon2011link}, one may also augment the model by incorporating some information regarding node affinities using implicit/explicit attributes and define node features $x_i$, which makes the model $\hat{G}_t(i,j) = \Phi(z_{i}^Tz_{j} + f_D(z_{i,j};w)$


%\subsubsection{Applications.} Our proposed technique can also be used in other applications. For example link recommendation and predicting missing edges in graphs. Or Vertex Recommendation similar to \cite{ou2016asymmetric} 

\subsubsection{Link privacy concern.}

%Connection to link privacy research such as \cite{amin:wwwj}

While link prediction techniques have a number of useful applications, it may increase the risk of link disclosure and thus increasing privacy concern. Even if the data owner removes sensitive links from the published network dataset, it may still be disclosed by link prediction and consequently lead to privacy breach. 

Michael et al. \cite{fire2013links} presented a link reconstruction attack, in which the attacker uses link prediction to infer a user's connections to others with high accuracy, but they did not mention how to defend the so-called link-reconstruction attack. Since link-reconstruction attack or Link prediction-based attack aims to find out some real but unobservable links, the defense of link-prediction-based attacks is also target-directed, which means that one has to preserve the targeted links from being predicted. Most existing approaches on link prediction are based on the similarity between nodes under the assumption that the more similar a pair of nodes are, the more likely a link exists between them.

Many algorithms have been developed for protecting the privacy of users, such as identity, relationship and attributes. In this paper, the focus is on preserving link privacy in social networks.

%In retrospect, Zheleva et al. \cite{zheleva2008preserving} proposed the concept of link re-identification attack, which refers to inferring sensitive relationships from anonymized network data. If the sensitive links can be identified by the released data, then this means privacy breach. Link perturbation is a common technique to preserve sensitive links. Zheleva et al. \cite{zheleva2008preserving} assumed that the adversary has an accurate probabilistic model for link prediction, and they proposed several heuristic approaches to anonymizing network data. Ying et al. \cite{ying2008randomizing} investigated the relationship between the level of link randomization and the possibility to infer the presence of a link in a network. Further, Ying et al. \cite{ying2009link} investigated the effect of link randomization on protecting privacy of sensitive links, and they found that similarity indices can be utilized by adversaries to significantly improve the accuracy in predicting sensitive links.

Zheleva et al. \cite{zheleva2008preserving} proposed the concept of link re-identification attack, which refers to inferring sensitive relationships from anonymized network data. If the sensitive links can be identified by the released data, then this means privacy breach. Link perturbation is a common technique to preserve sensitive links. 

\cite{zheleva2008preserving,ying2008randomizing} investigated the relationship between the level of link randomization and the possibility to infer the presence of a link in a network. Further, Ying et al. \cite{ying2009link} investigated the effect of link randomization on protecting privacy of sensitive links, and they found that similarity indices can be utilized by adversaries to significantly improve the accuracy in predicting sensitive links.
To avoid revealing the sensitive information about relationships, link privacy preserving systems provide a perturbed graph by adding extra noise to the network structure, e.g. \cite{hay2008resisting,mittalNDSS13,ying2008randomizing,zheleva2008preserving}. The challenge of preserving link privacy lies in causing no significant losses on the utility of applications that leverage the social trust relationships.

%Fard et al. [24] assumed that all links in a network are sensitive, and they proposed to apply subgraph-wise perturbations onto a directed network, which randomize the destination of a link within some subnetworks thereby limiting the link disclosure. Furthermore, they proposed neighborhood randomization to probabilistically randomize the destination of a link within a local neighborhood on a network \cite{amin:wwwj}. It should be noted that both subnetwork-wise perturbation and neighborhood randomization perturb every link in the network based on a certain probability.

%a data owner can add perturbations into the original network to reduce the risk of targeted-link disclosure due to link-prediction-based attacks


% visualize the network embeddings and features with t-SNE and use KL divergence to measure the performance



