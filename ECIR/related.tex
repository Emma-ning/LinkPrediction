\section{Related Work}

%Predicting Links in Multi-Relational and Heterogeneous Networks \cite{Yang2012} 

The problem of link prediction in static and homogeneous networks has been extensively studied in the past
\cite{liben2007link,wang2007local,lichtenwalter2010new,leroy2010cold,al2006link,al2011survey}, for which the probability of forming a link between two nodes is generally considered as a function of their topological similarity. However, such techniques cannot be directly applied to heterogeneous networks. A few works such as \cite{sun2011ASONAM,Sun:2012:HRP:2124295.2124373} investigated the problem of link prediction in HINs. Sun et al. \cite{sun2011ASONAM} showed that \textit{PathPredict} outperforms traditional link prediction approaches that use topological features defined on homogeneous networks such as common neighbors \cite{newman2001clustering}, preferential attachment \cite{newman2001clustering}, Jaccard's coefficient \cite{liben2007link}, and Katz$\beta$ \cite{katz1953new}. Different from the original link prediction problem, Sun et al. \cite{Sun:2012:HRP:2124295.2124373} studied the problem of predicting the time of relationship building in HINs. These works, however, do not consider the dynamism of networks and overlook the potential benefits of analyzing a HIN as a sequence of network snapshots.

%Recently researchers have focused on inferring latent space of networks for link prediction \cite{Zhu2016,ye2013predicting,qi2013link,dunlavy2011temporal,menon2011link} based on the assumption that the probability of a link between two nodes depends on their positions in their latent space. Each dimension of the latent space characterizes an attribute, and the more two nodes share such attributes the more likely they connect (also known as homophily). Amongst such graph embedding methods, a few  \cite{dunlavy2011temporal,Zhu2016} considered dynamic networks. Inspired by %the matrix factorization approach presented by 

Research works on static latent space inference of networks \cite{sarkar2005dynamic,menon2011link,yin2013scalable,qi2013link,ye2013predicting} have assumed that the latent positions of nodes are fixed, and only few graph embedding methods \cite{fu2009dynamic,dunlavy2011temporal,Zhu2016} have considered dynamic networks. Dunlavy et al. \cite{dunlavy2011temporal} developed a tensor-based latent space modeling to predict temporal links. Zhu et al. \cite{Zhu2016} added a temporal-smoothing regularization term to a non-negative matrix factorization objective to penalizes abrupt large changes in the latent positions and optimized it using a block-coordinate gradient descent algorithm. These works, however, do not consider heterogeneity of network structure.

 %Their formulation is almost identical to the algorithm of Chi et al. \cite{Chi:2007}, who perform evolutionary spectral clustering that captures temporal smoothness. Because matrix factorization provides embedding vectors of the nodes for each time-stamp, the factorization by-product from this work can be considered as dynamic network embeddings.

%Matrix factorization technique for link prediction, such as \cite{menon2011link},... Latent features are shown to be more predictive compared to unsupervised scoring techniques such as Katz, Adamic, and Preferential Attachment \cite{menon2011link,Zhu2016}. In our experiments we observed that combining latent with meta path-based features can increase prediction accuracy. However, if latent features learn similar structure as topological features do, then mixing them may not be beneficial. In such cases feature engineering techniques can be applied. 



%[12] D. Erdos, R. Gemulla, and E. Terzi, ?Reconstructing graphs from neighborhood data,? ACM Trans. Knowl. Discovery Data, vol. 8, no. 4, pp. 23:1?23:22, Oct. 2014.


%Such methods are homogeneous and non-temporal. The number of common neighbors \cite{newman2001clustering}, preferential attachment \cite{newman2001clustering}, Jaccard's coefficient \cite{liben2007link}, and Katz$\beta$ \cite{katz1953new}, are amongst frequently used topological features defined in homogeneous networks.

%Sun et al. proposed PathSelClus \cite{sun2013pathselclus} that uses limited guidance from users in the form of seeds in some of the clusters and automatically learn the best weights for each meta-path in the clustering process.

%The concept of temporal smoothness has been used in evolutionary clustering [36] and link prediction in a dynamic network \cite{Zhu2016}.

% In social networks, link prediction is used to predict probable friendships, which can be used for recommendation and lead to a more satisfactory user experience. Liben-Nowell et al. \cite{liben2007link}, Lu et al. [52] and Hasan et al. [53] survey the recent progress in this field and categorize the algorithms into (a) similarity based (local and global) [13], [14], [54], (b) maximum likelihood based [15], [16] and (c) probabilistic methods [17], [18], [55]. 
%Embeddings capture inherent dynamics of the network either explicitly or implicitly thus enabling application to link prediction. Wang et al. [23] and Ou et al. [24] predict links from the learned node representations on publicly available collaboration and social networks. In addition, Grover et al. [29] apply it to biology networks. They show that on these data sets links predicted using embeddings are more accurate than traditional similarity based link prediction methods described above.

% [13] P. Jaccard, Etude comparative de la distribution florale dans une portion des Alpes et du Jura. Impr. Corbaz, 1901.
% [14] L. A. Adamic and E. Adar, “Friends and neighbors on the web,” Social networks, vol. 25, no. 3, pp. 211–230, 2003.
% [15] A. Clauset, C. Moore, and M. E. Newman, “Hierarchical structure and the prediction of missing links in networks,” Nature, vol. 453,
% no. 7191, pp. 98–101, 2008.
% [16] H. C. White, S. A. Boorman, and R. L. Breiger, “Social structure from multiple networks. i. blockmodels of roles and positions,” American journal of sociology, vol. 81, no. 4, pp. 730–780, 1976.
% [17] N. Friedman, L. Getoor, D. Koller, and A. Pfeffer, “Learning probabilistic relational models,” in IJCAI, 1999, pp. 1300–1309.
% [18] D. Heckerman, C. Meek, and D. Koller, “Probabilistic entityrelationship models, prms, and plate models,” Introduction to statistical relational learning, pp. 201–238, 2007.
% [23] D. Wang, P. Cui, and W. Zhu, “Structural deep network embedding,” in Proceedings of the 22nd International Conference on Knowledge Discovery and Data Mining. ACM, 2016, pp. 1225–1234.
% [24] M. Ou, P. Cui, J. Pei, Z. Zhang, and W. Zhu, “Asymmetric transitivity preserving graph embedding,” in Proc. of ACM SIGKDD, 2016, pp. 1105–1114.
% [29] A. Grover and J. Leskovec, “node2vec: Scalable feature learning for networks,” in Proceedings of the 22nd International Conference on Knowledge Discovery and Data Mining. ACM, 2016, pp. 855–864.
% [52] L. Lu and T. Zhou, “Link prediction in complex networks: A survey,” Physica A: Statistical Mechanics and its Applications, vol.
% 390, no. 6, pp. 1150–1170, 2011.
% [53] M. Al Hasan and M. J. Zaki, “A survey of link prediction in social networks,” in Social network data analytics, 2011, pp. 243–275.
% [54] L. Katz, “A new status index derived from sociometric analysis,”
% Psychometrika, vol. 18, no. 1, pp. 39–43, 1953.
% [55] K. Yu, W. Chu, S. Yu, V. Tresp, and Z. Xu, “Stochastic relational models for discriminative link prediction,” in NIPS, 2006, pp. 1553–1560.