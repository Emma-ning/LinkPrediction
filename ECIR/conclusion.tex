\section{Conclusions and Future Work}

We studied the problem of relationship prediction in DHINs and proposed a supervised learning framework based on a combined set of latent and topological meta path-based features. Our results show that the proposed technique significantly improves prediction accuracy compared to the baseline methods. In this work we did not evaluate the running time and efficiency of our approach. Since our major computational bottleneck is calculating meta path-based measures, such as path count, we would like to investigate approximation techniques to make the prediction scalable. Furthermore we are interested in enhancing the matrix factorization technique based on a loss function that does not require full topological features matrix. In addition to model improvement, another interesting direction is to investigate the effectiveness of our proposed approach in other applications, such as predicting interests of users in a social media, that can be formulated as a link/relationship prediction problem.