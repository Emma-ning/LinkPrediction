\section{Introduction}
\label{Sec:Introduction}

% what is link prediction problem? what is the usages of link prediction?
% challenges in heterogeneous link prediction, what is not covered? temporal and hero?

The goal of link prediction in a network graph \cite{liben2007link} is to estimate the likelihood of future relationship between two nodes based on the observed graph. Predicting such connections in a network have multiple applications such as recommendation systems, network completion, or biomedical applications such as predicting protein-protein interactions. Traditional link prediction techniques, such as \cite{liben2007link}, consider networks to be homogeneous, i.e., graphs with only one type of nodes and edges. However, most real-world networks, such as social networks, scholar networks, patient networks \cite{denny2012mining} and knowledge graphs \cite{wang2015incorporating} are heterogeneous information networks (HINs) \cite{shi2017survey} and have multiple node and relation types. For example, in a bibliographic network there are nodes of types authors, papers, and venues, and edges of types writes, cites and publishes.%Research on HINs involves building a network schema and computing the semantic relatedness /similarities between two nodes along meta-paths \cite{sun2011pathsim}.

In a HIN relations between different entities carry different semantics. For instance the relationship between two authors are different in meaning when they are co-authors compared to the case that one cites another's paper. Thus techniques for homogeneous networks cannot be directly applied on heterogeneous ones. A few works such as \cite{sun2011pathsim,sun2011ASONAM} investigated the problem of link/relationship prediction in HINs, however, they do not consider the dynamism of networks and overlook the potential benefits of analyzing a heterogeneous graph as a sequence of network snapshots. To this end, existing work has already shown that in homogenous networks incorporating temporal changes improves link prediction accuracy \cite{Zhu2016}. Previous work on temporal link prediction scarcely studied HINs and to the best of our knowledge, the problem of relationship prediction for dynamic heterogeneous networks has not been studied before. A dynamic heterogeneous information network (DHIN) is a HIN, where links are associated with timestamps.

\amin{For Ebrahim}
In this work we study the problem of predicting relationships in a DHIN, which is stated as follows: \textit{Given a DHIN graph $G$ at time $t$, how can we predict the existence of a particular relationship/path between two given nodes at time $t+1$?}


%\cite{Zhu2016} \cite{sun2011pathsim} \cite{Sun:2012:HRP:2124295.2124373}  \cite{huang2016meta} \cite{wang2016relsim} \cite{sun2013pathselclus} \cite{sun2011ASONAM} \cite{Yang2012} \cite{liben2007link}


%\subsection{Motivation}
% what are the challenges in directly applying the current techniques?
% To address the above challenges, we propose a new ... First, , we introduce the concept of augmented graph, which allows us to incorporate more complex semantics into our prediction problem. Second, instead of computing the ...., we use all of the latent features of all meta-graphs. ...


%In \cite{gu2018rare} authors presented SocialRank...

The main contributions of our work include:

%The main objective of this work is to examine possibility of using meta path-based features along with latent features ... To this end, we propose to ... by taking into account (i) nodes? interest built based on their explicit contribution towards the extracted topics, (ii) theory of Homophily [12], which refers to the tendency of users to connect to users with common interests or preferences; and (iii) relationship between emerging topics, based on their similar constituent contents and user contributions towards them. More specifically, the key contributions of our work are as follows:

\begin{itemize}

\item We propose the problem of relationship prediction in a DHIN, and analyze the differences between this problem and the related link prediction solutions for dynamic networks and for heterogeneous networks;

\item We present a simple yet effective technique that leverages topological meta path-based and latent features to predict a target relationships between two nodes in a DHIN;

\item We define the \textit{augmented reduced graph} that is generated according to a given HIN and a target relation meta path and is used in our proposed algorithms: the homogenize link prediction, and the dynamic meta path-based relationship prediction;

%\item We implement our approach in a tool called ..., which is freely available

\item We empirically evaluate the efficacy and accuracy of our proposed algorithms on on two real-world datasets, and the results show X\% to Y\% improvement in accuracy  compared to baseline techniques.

\end{itemize}


In the rest of the paper, we introduce the preliminaries and problem statement in Section 2, discuss our solutions to the relationship prediction problem in Section 3, explain the details of our empirical experimentation and findings in Section 4, review the related work in Section 5, and finally conclude the paper.
