\section{Introduction}
\label{Sec:Introduction}

% what is link prediction problem? what is the usages of link prediction?

The goal of link prediction in a social network graph \cite{liben2007link} is to estimate the likelihood of the relationship between two nodes in future, based on the observed network. Predicting such connections in a network have multiple applications such as friend/item/ad recommending, network completion, or biological applications such as predicting protein-protein interactions.\amin{more citations} 

% challenges in heterogeneous link prediction, what is not covered? temporal and hero?

Traditional link prediction techniques \cite{liben2007link}\amin{more citations} consider networks to be homogeneous, i.e., graphs with only one type of edges and nodes. However, most real-world information networks, such as Twitter, Facebook, and DBLP, are heterogeneous and have multiple relation and node types. For example, in a bibliographic network there are nodes of types authors, papers, and venues, and edges of types write, cite and publish. There are limited number of works that focused on 
link prediction in heterogeneous information networks (HINs) such as \cite{sun2011pathsim}. However, those techniques do not consider the dynamics of social networks and ignore sequence of network snapshots. \amin{make sure about these citations} \cite{Zhu2016} \cite{sun2011pathsim} \cite{Sun:2012:HRP:2124295.2124373}  \cite{huang2016meta} \cite{wang2016relsim} \cite{sun2013pathselclus} \cite{sun2011ASONAM} \cite{Yang2012} \cite{liben2007link}

In this work we study the problem of predicting relationships in a dynamic heterogeneous information network (DHIN) i.e., a network with different types of nodes and links associated with timestamps, which is stated as follows: \textit{Given a DHIN graph G, how can we predict the future structure of G?}


\subsection{Motivation}

\amin{Motivate the problem by a real example and show the points made in this para.}

The link prediction problem for homogeneous networks has been studied in the past \cite{liben2007link}\amin{more citations}. However most real social networks are heterogeneous and relations between different entities have different semantic meanings. Thus techniques for homogeneous networks can not be directly applied on heterogeneous ones. A few works, such as \cite{sun2011pathsim,sun2011ASONAM}, investigated this problem, however, they do not consider the dynamics of social networks and ignore ignore sequence of network snapshots. On the other hand, it has ben shown that for homogenous network link prediction, incorporating temporal changes helps in a more accurate prediction \cite{Zhu2016}. Previous work on temporal link prediction scarcely studied heterogeneousness of social networks and to the best of our knowledge, the problem of relationship prediction for dynamic heterogeneous networks was not studied before.

%Large and complex databases, such as YAGO and DBLP, can be modeled as HINs. A fundamental problem in HINs is the computation of closeness, or relevance, between two HIN objects.

\subsection{Contributions}

The main contributions of our work include:

\begin{itemize}

\item We present a technique, called \texttt{RelationPredict}, that predicts a target relationships between two nodes of given types;

\item An evaluation of the accuracy and performance of the proposed algorithm on real social network data.

\end{itemize}


